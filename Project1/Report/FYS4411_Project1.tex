\documentclass[12pt,a4paper,english]{article}
\usepackage{times}
\usepackage[utf8]{inputenc}
\usepackage{babel,textcomp}
\usepackage{mathpazo}
\usepackage{mathtools}
\usepackage{amsmath,amssymb}
\usepackage{ dsfont }
\usepackage{listings}
\usepackage{graphicx}
\usepackage{ mathrsfs }
\usepackage{float}
\usepackage{subfig} 
\usepackage[hyphens]{url}
\usepackage[colorlinks]{hyperref}
\hypersetup{breaklinks=true}
\usepackage[usenames,dvipsnames,svgnames,table]{xcolor}
\usepackage{textcomp}
\definecolor{listinggray}{gray}{0.9}
\definecolor{lbcolor}{rgb}{0.9,0.9,0.9}
\lstset{backgroundcolor=\color{lbcolor},tabsize=4,rulecolor=,language=python,basicstyle=\scriptsize,upquote=true,aboveskip={1.5\baselineskip},columns=fixed,numbers=left,showstringspaces=false,extendedchars=true,breaklines=true,
prebreak=\raisebox{0ex}[0ex][0ex]{\ensuremath{\hookleftarrow}},frame=single,showtabs=false,showspaces=false,showstringspaces=false,identifierstyle=\ttfamily,keywordstyle=\color[rgb]{0,0,1},commentstyle=\color[rgb]{0.133,0.545,0.133},stringstyle=\color[rgb]{0.627,0.126,0.941},literate={å}{{\r a}}1 {Å}{{\r A}}1 {ø}{{\o}}1}

% Use for references
\usepackage[square,comma,numbers]{natbib}
%\DeclareRobustCommand{\citeext}[1]{\citeauthor{#1}~\cite{#1}}

% Fix spacing in tables and figures
%\usepackage[belowskip=-8pt,aboveskip=5pt]{caption}
%\setlength{\intextsep}{10pt plus 2pt minus 2pt}

% Change the page layout
%\usepackage[showframe]{geometry}
%\usepackage{layout}
\setlength{\hoffset}{-0.7in}  % Length left
\setlength{\voffset}{-0.7in}  % Length on top
\setlength{\textwidth}{480pt}  % Width /597pt
\setlength{\textheight}{660pt}  % Height /845pt
%\setlength{\footskip}{25pt}

\newcommand{\VEV}[1]{\langle#1\rangle}
\title{FYS4411 - Project 1}
\date{}
\author{ Kristoffer Langstad \footnote{\url{https://github.com/krilangs/FYS4411/tree/master/Project1} \cite{GitHub}}\\ \textit{krilangs@uio.no}}

\begin{document}%\layout
\maketitle
\begin{abstract}
	.................................
	MORE TO COME!!!
	.................................
\end{abstract}

\section{Introduction}
\label{sect:Introduction}
A more and more interesting theme in physics is to look at confined Bose systems. Systems like this contains gases of alkali atoms which are confined in magnetic traps which givens dilute systems. In such systems, we can study the Bose-Einstein condensation (BEC) when bosons are cooled down to a critical temperature. At this critical temperature, the bosons will condense into the system ground state. There we can study the ground state energy with a specific wave function. Most studies of the BEC confined in magnetic or optical traps have been conducted in the framework of the Gross-Pitaevskii (GP) equation \cite{nilsen2005vortices}.

In this project we will study two cases, one where the bosons don't interact with each other and one where they interact as if they had hard shells. For both cases we will study systems in one, two and three dimensions with a varying number of particles in the system. Since we are looking at bosons in the ground state of the system, we can derive the ground state energy analytically. Here the analytical expression is quite simple for non-interacting bosons. In the interacting case we need also a correlation wave function term which is not needed for non-interacting bosons. This will complicate the expression for the local energy. For the traps we will use spherical and elliptical harmonic traps. The calculated local energies for the systems we look at can also be calculated with the GP equation which we will use as a benchmark for our numerically calculated energies.

To do the ground state energy calculation for this many-body system, for both system cases, we will use Variational Monte Carlo (VMC) methods in C++ with both a "Brute force" Metropolis and a more improved Metropolis-Hastings algorithm, also called Importance sampling, and with a specific trial wave function. In this project we will use a single variational parameter $\alpha$. The optimal value of this variational parameter is found by implementing a gradient descent method for minimizing the ground state energy. With this optimal variational parameter we do a statistical analysis of the numerical data by using the blocking method to get a proper error analysis. Another thing we look at is the one-body density of the system where we study the correlation and importance of the Jastrow factor.

First we will look at the theory of the physics and the numerics we will use in this project. This includes an overview of among others the system, the analytical derivations of the local energy for both non-interacting and interacting cases, the Gross-Pitaevskii equation, one-body density and then the theory of the numerical methods mentioned above. In the methods section we explain the implementation of the numerical tools we are to use and how we will use them to do the calculations like the ground state energy in the theory section. The calculated ground state energies with statistical errors are in the end compared to the energies calculated with the GP equation in the three dimensional case. In the results section we will present and discuss all the results we get through this project. Lastly, we come up with a conclusion to the project with possible aspects towards future work.

\section{Theory}
\label{sect:Theory}
\subsection{The System}
\label{subsect:System}
We will consider a system of N bosons with masses $m$ which are trapped in a harmonic oscillator potential. For the alkali gas $^{87}$Rb, the characteristic dimensions/length of the trap is $a_{ho}=(\hbar/m\omega_{\perp})^{\frac{1}{2}}=1-2\times10^4$Å. For low energies it is sufficient enough to look at the s-wave scattering length $a_{Rb}$ which is often selected as $a_{Rb}=100a_0$, where $a_0=0.5292$Å is the Bohr radius. In \citet{dubois2001bose} they derive that both the trap size and the inter-atom spacing is much larger than the effective atom size leading to the condition for the system to be dilute. This leads to the system being dominated by two-body collisions. This gives that the system has the following two-body Hamiltonian
\begin{equation}
\label{eq:sys_Hamiltonian}
H=\sum_i^N\left(\frac{-\hbar^2}{2m}\nabla_i^2+V_{\text{ext}}(\textbf{r}_i)\right)+\sum_{i<j}^{N}V_{\text{int}}(\textbf{r}_i,\textbf{r}_j),
\end{equation}
which is going to work on a trial wave function $\Psi_T(\textbf{r})$. We will look at both a spherical (S) and an elliptical (E) harmonic trap up till three dimensions. They are modeled as a hard-shell models. The $V_{\text{ext}}(\textbf{r}_i)$ term is the harmonic oscillator potential, which in three dimensions is given by
\begin{equation}
\label{eq:V_ext}
V_{\text{ext}}(\textbf{r})= \Bigg\{
\begin{array}{ll}
\frac{1}{2}m\omega_{ho}^2r^2 & (S)\\
\strut
\frac{1}{2}m[\omega_{ho}^2(x^2+y^2) + \omega_z^2z^2] & (E)
\end{array}.
\end{equation}
Here $\omega_{ho}^2$ is the trap potential strength. For elliptical trap, $\omega_{ho}=\omega_{\perp}$ is the trap frequency in the perpendicular plane, while $\omega_z$ is the frequency in the z-direction. The ratio of the frequencies is denoted as $\lambda=\omega_z/\omega_{\perp}$ giving the ratio of the trap lengths 
\begin{equation*}
\frac{a_{\perp}}{a_z}=\left(\frac{\omega_z}{\omega_{\perp}}\right)^{\frac{1}{2}}=\sqrt{\lambda}.
\end{equation*} 
The $V_{\text{int}}(r_{ij})$ term is the hard-shell interaction potential given as
\begin{equation}
\label{eq:V_int}
V_{\text{int}}(r_{ij})= \Bigg\{
\begin{array}{ll}
\infty & |\textbf{r}_i-\textbf{r}_j|\leq a\\
\strut
0 & |\textbf{r}_i-\textbf{r}_j|> a
\end{array},
\end{equation}
with $r_{ij}=|\textbf{r}_i-\textbf{r}_j|$ is the distance between particle $i$ and $j$, and $a$ is the so-called hard-core diameter of the bosons.

For the ground state with N particles we will us the trial wave function
\begin{equation}
\label{eq:WF_T}
 \Psi_T(\mathbf{r})=\Psi_T(\mathbf{r}_1, \mathbf{r}_2, \dots \mathbf{r}_N,\alpha,\beta)
=\left[
\prod_i g(\alpha,\beta,\mathbf{r}_i)
\right]
\left[
\prod_{j<k}f(a,|\mathbf{r}_j-\mathbf{r}_k|)
\right], 
\end{equation}
where $\alpha$ and $\beta$ are variational parameters. We will use $\alpha$ as the only variational parameter, and treat $\beta$ as a constant to be chanegd according to if we have a spherical trap ($\beta=1$) or an elliptical trap ($\beta=2.82843$). The first part of the trial wave function (\ref{eq:WF_T}) is the one body harmonic oscillator ground state wave function, also called the Slater permanent,
\begin{equation}
\label{eq:g}
g(\alpha,\beta,\mathbf{r}_i)= e^{-\alpha(x_i^2+y_i^2+\beta z_i^2)},
\end{equation} 
and the second part is the correlation wave function, also called the Jastrow factor. For spherical traps and non-interaction bosons ($a=0$) we get $\alpha=\frac{1}{2a_{ho}^2}$ the correlation term yields
\begin{equation}
\label{eq:Jastrow}
J(\textbf{r})=f(a,\mathbf{r}_i,\mathbf{r}_j)=\Bigg\{
\begin{array}{ll}
0 & {|\mathbf{r}_i-\mathbf{r}_j|} \leq {a}\\
(1-\frac{a}{|\mathbf{r}_i-\mathbf{r}_j|}) & {|\mathbf{r}_i-\mathbf{r}_j|} > {a}.
\end{array}
\end{equation}
When scaling and using natural units later in this project, $\hbar=\omega=m=1$, we get $a_{ho}=1$ (as can be seen easily above) which then gives $\alpha=0.5$ for the ground state.

\subsection{Local Energy}
\label{subsect:Energy}
In this project, the quantity we will mostly be looking at is the local energy of the system defined as
\begin{equation}
\label{eq:local_energy}
E_L(\mathbf{r})=\frac{1}{\Psi_T(\mathbf{r})}H\Psi_T(\mathbf{r}).
\end{equation}  
The calculation of the energy of the system is done by computing the expectation value of the local energy:
\begin{equation}
\label{eq:expec_E}
\langle E_L\rangle=\langle H\rangle=\frac{\int E_L(\textbf{r})|\Psi_T(\textbf{r})|^2d\textbf{r}}{\int|\Psi_T(\textbf{r})|^2d\textbf{r}}\approx\frac{1}{N}\sum_{i=1}^{N}E_L(\textbf{r}_i)|\psi(\textbf{r}_i)|^2
\end{equation} 
$|\psi(\textbf{r})|^2$ is the probability density function. The ground state energy is where we have the minimum of the expectation value of the energy. To determine this we also calculate the variance 
\begin{equation}
\label{eq:variance}
\sigma^2=\langle E_L^2\rangle-\langle E_L\rangle^2,
\end{equation}
and the standard deviation $\sigma$. We want the variance to be as small as possible.

To compute the local energy we will find analytical expressions and compare it with numerically calculated energies of the system. Here we expect the analytical expression to be much more CPU friendly.

\subsection{Drift Force}
\label{subsect:F}
Since the brute force Metropolis-Hastings algorithm might use more time than needed at unwanted places, we want improve how the algorithm makes moves. In other words, we want to increase the acceptance rate of the proposed moves in the algorithm. So we introduce Importance sampling with a drift/quantum force $F(\textbf{r})$. The drift force is derived from the Fokker-Planck equation:
\begin{equation}
\label{eq:Fokker-Planck}
\frac{\partial P(\textbf{r},t)}{\partial t}=\sum_{k}D\left(\nabla_k^2\cdot P(\textbf{r},t)-D\nabla_k\cdot F(\textbf{r}_k)P(\textbf{r},t)\right)=\sum_{k}D\nabla_k\cdot(\nabla_k-F(\textbf{r}_k))P(\textbf{r},t)
\end{equation}
The first part is the standard diffusion equation with a diffusion coefficient $D$, and $P(\textbf{r},t)$ is a time-dependent probability density. We want a convergence to a stationary state with
\begin{align*}
\frac{\partial P}{\partial t}=0.
\end{align*}
This is only true if
\begin{align*}
\nabla_k^2 P=P\nabla_k \textbf{F}_k+\textbf{F}_k\nabla_k P.
\end{align*}
With the assumption of stationary densities:
\begin{align*}
\textbf{F}&=g(\textbf{r})\nabla P\\
\Rightarrow\quad \nabla_k^2 P&=P\nabla_k g\cdot(\nabla_k P)^2+Pg\nabla_k^2 P+g(\nabla_k P)^2
\end{align*}
The terms containing first and second derivatives have to cancel each other, which only happens if $g=\frac{1}{P}$. For a system with bosons the trial wave function is the probability density function and $D=1/2$ comes from the half factor in the kinetic energy (natural units). This yields the drift force for a single article $k$:
\begin{equation}
\label{eq:F}
F_k(\textbf{r})=\frac{2\nabla_k\Psi_T(\textbf{r})}{\Psi_T(\textbf{r})}
\end{equation}
This force will push the proposed moves towards the regions of configuration space where the trial wave function is large, which should increase the acceptance rate of the moves made.

\subsection{The Gross-Pitaevskii equation}
\label{subsect:GP_eq}
In this project we look at alkali gases which are diluted. This means that they can be computed using the mean field Gross-Pitaevskii (GP) equation. The GP equation in \citet{dubois2001bose} is defined as
\begin{equation}
\label{eq:GP_eq}
E_{GP}[\Psi]=\int d\textbf{r}\left[\frac{\hbar^2}{2m}|\nabla\Psi(\textbf{r})|^2+V_{\text{trap}}(\textbf{r})|\Psi|^2+\frac{2\pi\hbar^2a}{m}|\Psi|^4\right],
\end{equation}
where $V_{\text{trap}}(\textbf{r})=\frac{1}{2}m(\omega_{\perp}^2x^2+\omega_{\perp}^2y^2+\omega_z^2z^2)$ is the three dimensional elliptical harmonic potential. For a non-interacting system with $a=0$ for N particles we use with a normalized trial wave function 
\begin{align}
\label{eq:norm_WF_T}
\Psi_T&=\left(\frac{2\alpha}{\pi}\right)^{\frac{3N}{4}}\beta^{\frac{N}{4}}\prod_{i}e^{-\alpha(x_i^2+y_i^2+\beta z_i^2)}\\
\nabla\Psi_T&=\left(\frac{2\alpha}{\pi}\right)^{\frac{3N}{4}}\beta^{\frac{N}{4}}\prod_{i}e^{-\alpha(r_i^2)}\cdot\sum_i(-2\alpha(x_i\textbf{e}_x+y_i\textbf{e}_y+\beta z_i\textbf{e}_z))
\end{align}
The GP equation (\ref{eq:GP_eq}) can then be rewritten to
\begin{align}
\label{eq:GP_derived}
E[\Psi_T]&=\sum_i\int(2\alpha^2+\frac{1}{2})(x_i^2+y_i^2+\beta z_i^2)|\Psi_T|^2d\textbf{r}\nonumber\\
&=(2\alpha^2+\frac{1}{2})\left(\frac{2\alpha}{\pi}\right)^{\frac{3N}{2}}\beta^{\frac{N}{2}}\frac{N}{4\alpha}(\beta+2)\left(\frac{\pi}{2\alpha}\right)^{\frac{3N}{2}}\beta^{-\frac{N}{2}}\nonumber\\
&=\frac{N}{2}\left(\alpha+\frac{1}{4\alpha}\right)(\beta+2).
\end{align}
This equation (\ref{eq:GP_derived}) is the benchmark energy to be used for comparison to the computed local energy for spherical and elliptical harmonic oscillator traps in three dimensions for N particles later in the project.

\subsection{Non-Interacting Bosons}
\label{subsect:non-int}
\subsubsection{Drift Force and Local Energy Analytical}
\label{subsubsect:EL_analytic}
For non-interacting bosons in a spherical trap with $a=0$ and $\beta=1$, the correlation term in the trial wave function (eq. \ref{eq:WF_T}) vanishes:
\begin{equation}
\label{eq:nonint_WF_T}
\Psi_T(\textbf{r})=\prod_i g(\alpha,1,\mathbf{r}_i)=\prod_{i}e^{-\alpha(x_i^2+y_i^2+z_i^2)}=\prod_{i}e^{-\alpha|\textbf{r}_i|^2}
\end{equation}
In this case it is easy to switch between the number of dimensions since 1D: $\textbf{r}_i^2=x_i^2$, 2D: $\textbf{r}_i^2=x_i^2+y_i^2$ and 3D: $\textbf{r}_i^2=x_i^2+y_i^2+z_i^2$.

Since $a=0$ the interaction potential $V_{\text{int}}(\textbf{r}_{ij})$ (eq. \ref{eq:V_int}) also vanishes, and we get the spherical term of the harmonic potential $V_{\text{ext}}(\textbf{r})$ (eq. \ref{eq:V_ext}). This then gives the new Hamiltonian as
\begin{equation}
\label{eq:non-int_H}
H=\sum_{i=1}^N\left(\frac{-\hbar^2}{2m}\nabla_i^2+\frac{1}{2}m\omega_{ho}^2r_i^2\right).
\end{equation}
We now have the trial wave function and the Hamiltonian needed for the local energy in equation \ref{eq:local_energy}. The next thing we have to compute is the gradient ($\nabla$) and the Laplacian ($\nabla^2$) of the trial wave function. The gradient for a particle $k$ of equation \ref{eq:nonint_WF_T} is
\begin{equation}
\label{eq:grad_WF_T}
\nabla_k\Psi_T(\textbf{r})=-2\alpha \textbf{r}_ke^{-\alpha|\textbf{r}_k|^2}=-2\alpha \textbf{r}_k\Psi_T(\textbf{r}).
\end{equation} 
The Laplacian is then
\begin{equation}
\label{eq:Laplacian_WF_T}
\nabla_k^2\Psi_T(\textbf{r})=(-2\alpha d+4\alpha^2r_k^2)e^{-\alpha|\textbf{r}_k|^2}=(-2\alpha d+4\alpha^2r_k^2)\Psi_T(\textbf{r}),
\end{equation}
with dimensionality $d$ of the system.

For a particle $k$ we can use the gradient of the trial wave function (eq. \ref{eq:grad_WF_T}) to find the drift force from equation \ref{eq:F}:
\begin{equation}
\label{eq:Drift_F_nonint}
F_k(\textbf{r})=\frac{2}{\Psi_T(\textbf{r})}(-2\alpha \textbf{r}_k)\Psi_T(\textbf{r})=-4\alpha\textbf{r}_k
\end{equation}

With the Laplacian of the trial wave function in equation \ref{eq:Laplacian_WF_T} we can compute the non-interacting Hamiltonian in equation \ref{eq:non-int_H} and then the local energy in equation \ref{eq:local_energy}:
\begin{align*}
E_L(\textbf{r})&=\frac{1}{\Psi_T(\textbf{r})}\sum_{i=1}^{N}\left(\frac{-\hbar^2}{2m}(-2\alpha d+4\alpha^2r_i^2)\Psi_T(\textbf{r})+\frac{1}{2}m\omega_{ho}^2r_i^2\Psi_T(\textbf{r})\right)\nonumber\\
&=\sum_{i=1}^{N}\left(\frac{-\hbar^2}{2m}(-2\alpha d+4\alpha^2r_i^2)+\frac{1}{2}m\omega_{ho}^2r_i^2\right)
\end{align*}
Since we will use scaling and natural units, $\hbar=m=c=1$ and use $\omega_{ho}=1$, we can simplify the local energy to
\begin{equation}
\label{eq:local_E_nonint}
E_L(\textbf{r})=\alpha dN+(-2\alpha^2+\frac{1}{2})\sum_{i=1}^{N}r_i^2.
\end{equation}
If we choose $\alpha=0.5$; 
\begin{equation}
\label{eq:local_E_stable}
E_L(\textbf{r};\alpha=0.5)=\frac{dN}{2},
\end{equation}
we see that the local energy becomes stable depending on the dimensionality of the system and the number of particles. This will give the ground state energy of the system in this case. In appendix \ref{appendix:Optimal_alpha} we derive another way to show that $\alpha=0.5$ is the optimal variational parameter in this case.

\subsubsection{Local Energy Numerical}
\label{subsubsect:EL_num}
We will also in this non-interacting case compute the local energy numerically. Here we calculate the kinetic and potential energy and sum them up. With scaling, natural units and oscillator frequency $\omega=\omega_{ho}$:
\begin{equation}
\label{eq:local_E_num}
E_L=E_k+E_p=-\frac{1}{2}\frac{1}{\Psi_T}\sum_i^{N}\nabla_i^2\Psi_T+\frac{1}{2}\sum_{i=1}^{N}\omega^2r_i^2
\end{equation}
To solve this system numerically we use the Forward Euler finite difference approximation method, with time step $h$, for each dimension (x, y, z):
\begin{equation*}
\nabla^2 u\approx \frac{u(i+h)-2u(i)+u(i-h)}{h^2},\quad i=0,1,...,N
\end{equation*}

\subsection{Interacting Bosons}
\label{subsect:interact}
\subsubsection{Drift Force and Local Energy}
\label{subsubsect:local_E_int}
For the more realistic system we look at interacting bosons in the potential trap. For the interacting elliptical system $a\neq0$ and $\beta\neq0$. Now we have the full trial wave function in equation \ref{eq:WF_T}. Then we change the Slater permanent (eq. \ref{eq:g}) as $g(\alpha,\beta,\mathbf{r}_i)=\phi(\textbf{r}_i)$, and the Jastrow factor (eq. \ref{eq:Jastrow}) as 
\[\prod_{j<k}f(r_{jk})=\exp\left(\sum_{j<k}u(r_{jk})\right)\]
with $r_{jk}=|\textbf{r}_j-\textbf{r}_k|$ and $u(r_{jk})=\ln f(r_{jk})$. The final expression for the trial wave function is then
\begin{equation}
\label{eq:WF_T_int}
\Psi_T(\mathbf{r})=\left[\prod_i \phi(\textbf{r}_i)\right]
\exp\left(\sum_{j<k}u(r_{jk})\right).
\end{equation}

To find the drift force (eq. \ref{eq:F}) for particle $k$ in the interacting case, we take the gradient of this new trial wave function (also used in the Laplacian later): (Note that we have to change the indices since they are different)
\begin{align}
\label{eq:grad_int}
\nabla_k\Psi_T(\textbf{r})&=\nabla_k\left[\prod_i \phi(\textbf{r}_i)
\exp\left(\sum_{j<m}u(r_{jm})\right)\right]\nonumber\\
&=\nabla_k\phi(\textbf{r}_k)\left[\prod_{i\neq k}\phi(\textbf{r}_i)\right] \exp\left(\sum_{j<m}u(r_{jm})\right) +\left[\prod_{i}\phi(\textbf{r}_i)\right]\nabla_k\left[\exp\left(\sum_{j<m}u(r_{jm})\right)\right]\nonumber\\
&=\nabla_k\phi(\textbf{r}_k)\left[\prod_{i\neq k}\phi(\textbf{r}_i)\right] \exp\left(\sum_{j<m}u(r_{jm})\right) +\left[\prod_{i}\phi(\textbf{r}_i)\right]\exp\left(\sum_{j<m}u(r_{jm})\right)\sum_{l\neq k}\nabla_k u(r_{kl})\nonumber\\
&=\frac{\nabla_k\phi(\textbf{r}_k)}{\phi(\textbf{r}_k)}\Psi_T(\textbf{r})+\Psi_T(\textbf{r})\sum_{l\neq k}\nabla_k u(r_{kl})
\end{align}
The drift force for particle $k$ then becomes:
\begin{align}
\label{eq:drift_force_int}
F_k({\textbf{r}})&=\frac{2}{\Psi_T(\textbf{r})}\left(\frac{\nabla_k\phi(\textbf{r}_k)}{\phi(\textbf{r}_k)}+\sum_{l\neq k}\nabla_k u(r_{kl})\right)\Psi_T(\textbf{r})\nonumber\\
&=2\left(\frac{\nabla_k\phi(\textbf{r}_k)}{\phi(\textbf{r}_k)}+\sum_{l\neq k}\nabla_k u(r_{kl})\right)
\end{align}

The next thing needed again is to compute the second derivative in the full Hamiltonian (eq. \ref{eq:sys_Hamiltonian}). The Laplacian divided by the trial wave function becomes:
\begin{align}
\label{eq:Laplacian_int}
\frac{\nabla_k^2\Psi_T(\textbf{r})}{\Psi_T(\textbf{r})}&= \frac{\nabla_k^2\phi(\textbf{r}_k)}{\phi(\textbf{r}_k)}+2\frac{\nabla_k\phi(\textbf{r}_k)}{\phi(\textbf{r}_k)}\left(\sum_{j\neq k}u^{\prime}(r_{kj})\frac{\Delta\textbf{r}_{kj}}{r_{kj}}\right)+\sum_{i,j\neq k}\frac{\Delta \textbf{r}_{ki}\Delta \textbf{r}_{kj}}{r_{ki}r_{kj}}u^{\prime}(r_{ki})u^{\prime}(r_{kj})\nonumber\\
&+\sum_{j\neq k}\left(u^{\prime\prime}(r_{kj})+\frac{d-1}{r_{kj}}u^{\prime}(r_{kj})\right)
\end{align}
The full derivation of this equation can be seen in Appendix \ref{appendix:Laplacian}, and $d$ is the dimensionality of the system. For the single-particle function with interaction we evaluate the gradient and the Laplacian:
\begin{align}
\label{eq:eval_grad_one}
\frac{\nabla_k\phi(\textbf{r}_k)}{\phi(\textbf{r}_k)}&=-2\alpha(x_k\textbf{e}_x+y_k\textbf{e}_y+\beta z_k\textbf{e}_z)\\
\label{eq:eval_laplacian_one}
\frac{\nabla_k^2\phi(\textbf{r}_k)}{\phi(\textbf{r}_k)}&=-2\alpha(d-1+\beta)+4\alpha^2(x_k^2+y_k^2+\beta^2z_k^2)
\end{align}
Then the same for the correlation function (eq. \ref{eq:Jastrow}):
\begin{align}
\label{eq:eval_grad_int}
u^{\prime}(r_{kj})&=\frac{\partial u(r_{kj})}{\partial r_{kj}}=\frac{a}{r_{kj}(r_{kj}-a)}\\
\label{eq:eval_laplacian_int}
u^{\prime\prime}(r_{kj})&=\frac{\partial^2 u(r_{kj})}{\partial r_{kj}^2}=\frac{a^2-2ar_{kj}}{r_{kj}^2(r_{kj}-a)^2}
\end{align}
Inserting all the equations (\ref{eq:Laplacian_int})-(\ref{eq:eval_grad_int}) with the Hamiltonian for interacting bosons (eq. \ref{eq:sys_Hamiltonian}) into the local energy (eq. \ref{eq:local_energy}), we get (full derivation in Appendix \ref{appendix:local_E}):
\begin{align}
\label{eq:local_E_full}
E_L(\textbf{r})=-\frac{\hbar^2}{2m}\sum_k^N\frac{\nabla_k^2\Psi_T(\textbf{r})}{\Psi_T(\textbf{r})}
+\frac{1}{2}m\sum_k^N[\omega^2(x_k^2+y_k^2) + \omega_z^2z_k^2]
+\sum_{j<k}^{N}V_{\text{int}}(\textbf{r}_j,\textbf{r}_k)
\end{align}
Insert the Laplacian divided by the trial wave function in equation \ref{eq:H_int_deriv} into the local energy above for full expression. We do not write the full expression since the equation becomes so large.

\subsubsection{Scaling}
\label{subsubsect:Scaling}
For the elliptic trap with a repulsive interaction we introduce energy in units of $\hbar\omega_{ho}$ and lengths in units of $a_{ho}=\sqrt{\frac{\hbar}{m\omega_{ho}}}=(1-2)\times10^4$Å such that $r\rightarrow r/a_{ho}$. We also fix the scaled hard-core radius $a/a_{ho}=0.0043$. The original Hamiltonian in three dimensions for N particles can then be rearranged as
\begin{align*}
H&=\sum_{k=1}^N\left(\frac{-\hbar^2}{2m}\nabla_k^2+\frac{1}{2}m\left[\omega_{ho}^2(x_k^2+y_k^2)+\omega_z^2z_k^2\right]\right)+\sum_{k<j}^{N}V_{\text{int}}(\textbf{r}_k,\textbf{r}_j)\\
&=\sum_{k=1}^N\frac{\hbar\omega_{ho}}{2}\left(\frac{-\hbar}{m\omega_{ho}}\nabla_k^2+\frac{m\omega_{ho}}{\hbar}\left[x_k^2+y_k^2+\frac{\omega_z^2}{\omega_{ho}^2}z_k^2\right]\right)+\sum_{k<j}^{N}V_{\text{int}}(\textbf{r}_k,\textbf{r}_j)
\end{align*}
Now by changing to the mentioned units, meaning that we make the energy and length dimensionless, and introduce $\gamma=\frac{\omega_z}{\omega_{ho}}$, we get a dimensionless Hamiltonian:
\begin{equation}
\label{eq:H_scaled}
H=\sum_{k=1}^N\frac{1}{2}\left(-\nabla_k^2+x_k^2+y_k^2+\gamma^2z_k^2\right)+\sum_{k<j}^{N}V_{\text{int}}(\textbf{r}_k,\textbf{r}_j)
\end{equation}

\subsection{Statistical Error Analysis}
\label{subsect:Error_analysis}
An important part of the results we get are errors is these results. In most cases the final results will have some errors that have to be considered to actually give the final right results. These errors can be categorized into statistical and systematical errors. Systematical errors are method dependent, mostly due to some changing error in often the method used or the machinery producing the data. This means that they have to be considered differently from case to case. The statistical errors are the differences between the value resulting from the experiment and the known exact solution, which in most cases actually is unknown. To calculate the statistical errors we use the central limit theorem which gives that the more data points we use, the closer our resulting distribution will be to the normal distribution. To use this theorem, we have to make an assumption that we have independent and identically distributed (iid) events. With the probability distribution $p(x)$ for $n$ measured data points, we can then calculate the mean value
\begin{equation}
\label{eq:stat_mean}
\langle x \rangle =\mu=\frac{1}{n}\sum_{i=1}^{n}p(x_i)x_i.
\end{equation}
With this mean value, we can calculate the sample variance 
\begin{equation}
\label{eq:stat_ar}
\sigma^2=\langle x^2 \rangle-\mu^2=\frac{1}{n}\sum_{i=1}^{n}p(x_i)(x_i-\mu)^2,
\end{equation}
and the standard deviation
\begin{equation}
\label{eq:stat_STD}
\text{STD}=\frac{\sigma}{\sqrt{n}}.
\end{equation}
For iids there should not be any correlations between the data points. In this project we use a random generator to make produce the data points. These random generators do not give 100\% uncorrelated data points, which means that the sample variance gets a covariance term as well (also called the sample error)
\begin{equation}
\label{eq:stat_var_cov}
\sigma_m^2=\frac{\sigma^2}{n}+\text{cov}.
\end{equation}

Computationally we will use Monte-Carlo calculations, which runs the experiment M number of samples/MC cycles. To account for the possible numerical precision loss we introduce a correlation function 
\begin{equation}
\label{eq:stat_corr_func}
f_d=\frac{1}{n-d}\sum_{k=1}^{n-d}(x_k-\langle x_n\rangle)(x_{k+d}-\langle x_n\rangle),
\end{equation}
where $d$ is the distance between the measurements in the sample samples. This now yields an autocorrelation function
\begin{equation}
\label{eq:stat_autocorr}
\kappa_d=\frac{f_d}{\text{Var}(x)}
\end{equation}
giving value 1 if there are no correlation between the data. The sample error yields
\begin{equation*}
\sigma_m^2=\frac{\sigma^2}{n}+\frac{2}{n}\sigma^2\sum_{d=1}^{n-1}\frac{f_d}{\sigma^2}.
\end{equation*}
Now we introduce a correction factor $\tau$ which we will call the autocorrelation time
\begin{equation}
\label{eq:stat_autocorr_time}
\tau= 1+2\sum_{d=1}^{n-1}\kappa_d.
\end{equation}
To find this $\tau$ factor, we will use what is called a blocking method to be more explained later.

\subsection{One-Body Density}
\label{subsect:One_body_density}
A good characterization of where the particles are most likely to exist in the ground state of the system is the single-particle density function, also called the one-body density $\rho(\textbf{r})$. This density represents the distribution of all the particles in the system, or the probability of finding a particle at a distance $r$ from the origin. It is defined by the integral of the trial wave function over all the other coordinates except the one we are looking at:
\begin{equation}
\label{eq:one-body}
\rho(\textbf{r}_1)=\int |\Psi_T(\textbf{r}_1,\cdots,\textbf{r}_N)|^2d\textbf{r}_2 \cdots d\textbf{r}_N
\end{equation}
The one-body density is then normalized such that 
\[N=\int\rho(\textbf{r}_i)d\textbf{r}_i.\]
This means that the integral over the entire volume in the direction of any particle $i$ returns the total number of particles in the system.

\section{Numerical Algorithms}
\label{sect:Num_algos}
\subsection{Variational Monte Carlo}
\label{subsect:VMC}
To do the numerics in this project we will use a well known numerical method called variational Monte Carl (VMC). The good thing about this method is that we do not need to deal with differential equations which usually are difficult to handle. Instead we use statistical simulation to describe a system. This method uses random sampling of probability distributions through usually a large number of steps to achieve a solution. This type of Monte Carlo simulation needs some sampling technique combined with with an algorithm for accepting or declining a proposed move towards the most probable result. In this project we will apply a two variants of the Metropolis algorithm; a more brute force Metropolis and a improved Metropolis-Hastings/Importance Sampling algorithm. 

Most quantum systems are many-body problems with a large amount of particles. For such systems, it is often too difficult to obtain exact solutions. That is why the quantum Monte Carlo method is very good for simulating such systems, since we can compute different expectation values like the ground state energy. It is this quantity we will focus on in this project for a system of N particles up till three dimensions. We will utilize the variational principle for a given trial wave function $\Psi_T$ and Hamiltonian states that the expectation value of the Hamiltonian will be an upper bound to the ground state energy $E_0$ of the system as:
\begin{equation}
\label{eq:variational_prin}
E_0\leq E[H]=\frac{\int\Psi_T^*\hat{H}\Psi_Td\textbf{r}}{\int\Psi_T^*\Psi_T d\textbf{r}}
\end{equation}
To determine if we have located a minimum we use the variance of the expected energy as in equation \ref{eq:variance}.

The MC integration is used to avoid the multi-dimensional integral of a expectation value by approximating it to a sum over the number of MC cycles:
\begin{equation*}
\langle x\rangle =\int xp(x)dx\approx\frac{1}{MC}\sum_{i=1}^{MC}x_ip(x_i)
\end{equation*}
From the law of large numbers we would then expect
\[\lim\limits_{MC\rightarrow\infty}\frac{1}{MC}\]
To use the MC integration we need a probability distribution, which we will choose as 
\[p(\textbf{r})=|\Psi_T(\textbf{r})|^2.\]

This VMC scheme will first initialize the trial wave function, which is dependent on one variational parameter $\alpha$ (in this project) that we want as close as possible to the optimal parameter. Then we initialize the position of the particles, the energy and variance. For a given number of MC cycles we start the MC simulation:
\begin{enumerate}
	\item Here we will propose a new step $R^{\prime}=R+r\delta$, where $r\in[-0.5,0.5]$ is drawn at random and $\delta$ is tuned. 
	\item Then we calculate the trial wave function and the probability distribution function of the new position.
	\item Use the Metropolis to test if we should accept or decline the new position.
	\item For accepted position we update the initial position to the proposed step.
	\item Then we update the energy and variance.
\end{enumerate} 
After the MC simulation we compute the average of the interesting quantities. 

\subsection{Metropolis Algorithms}
\label{subsect:Metropolis}
The Metropolis algorithm uses the fundamentals of Markov Chains for a random walk to evolve the probability distribution function from a state $j$ to a state $i$. This transition is described by the transition probability matrix $W(j\rightarrow i)=W_{j\rightarrow i}$ and is not usually known. This transition matrix can be modeled into a multiplication of two parts; $A(j\rightarrow i)$ which is the probability of accepting a proposed move from $j$ to $i$, and $T(j\rightarrow i)$ which is the likelihood for the transition from $j$ to $i$ to happen: 
\begin{equation}
\label{eq:Transition_prob}
W(j\rightarrow i)=A(j\rightarrow i)\cdot T(j\rightarrow i)
\end{equation}
This leads to the probability of the system to occur in the state $i$ after $n$ steps to be:
\begin{equation*}
\label{eq:Prob_state1}
P_i^{(n)}=\sum_{j}\left[P_j^{(n-1)}T_{j\rightarrow i}A_{j\rightarrow i}+ P_i^{(n-1)}T_{i\rightarrow j}(1-A_{i\rightarrow j})\right]
\end{equation*}
The probability of making some transition must be one such that $\sum_{j}T_{j\rightarrow i}=1$, which yields
\begin{equation*}
\label{eq:Prob_state2}
P_i^{(n)}=P_i^{(n-1)}+\sum_{j}\left[P_j^{(n-1)}T_{j\rightarrow i}A_{j\rightarrow i}- P_i^{(n-1)}T_{i\rightarrow j}A_{i\rightarrow j}\right]
\end{equation*}
For large $n$ we require that $P_j^{(n\rightarrow\infty)}=p_i$. This gives what is called the balance requirement
\[\sum_{j}\left[p_jT_{j\rightarrow i}A_{j\rightarrow i}+ p_iT_{i\rightarrow j}(1-A_{i\rightarrow j})\right],\]
which can be rewritten to a more sturdy requirement called the detailed balance:
\begin{equation}
\label{eq:detailed_balance}
\frac{A_{j\rightarrow i}}{A_{i\rightarrow j}}=\frac{p_iT_{i\rightarrow j}}{p_jT_{j\rightarrow i}}
\end{equation}
This requirement is what decides if the new step is accepted or not. This is the foundation of the two Metropolis methods we are using in this project.

\subsubsection{Brute Force Metropolis}
\label{subsubsect:brute_Metropolis}
The first thing in the brute force Metropolis algorithm is that we assume the transition likelihood to be symmetric, $T_{j\rightarrow i}=T_{i\rightarrow j}$. Since this is assumed to be symmetric and the acceptance probabilities are unknown, the acceptance ratio is defined as the ratio of the different probabilities. We want the highest possible number of accepted moves in number of proposed steps, which means we want to maximize 
\[A_{j\rightarrow i}=\min(1, \frac{p_i}{p_j}).\] 
The acceptance ratio in equation \ref{eq:detailed_balance} will have the following effect: If the ratio is equal to 1 the proposed move is stand still, if the ratio is larger than 1 the proposed move is moving towards the most probable state and if the ratio is smaller than 1 the proposed move is moving away from the most probable state. So for a random number $r\in[-0.5,0.5]$ (uniformly distributed) we will accept the move if
\begin{equation}
\label{eq:brute_accept_move}
r\leq\frac{|\Psi_T(\textbf{r}_{k+1})|^2}{|\Psi_T(\textbf{r}_k)|^2},
\end{equation}
and for each MC cycle the new position is proposed \[R^{\prime}=R+r\delta\]
where $\delta$ is tuned to give acceptance rate of $\approx50$\% of the proposed moves. Then the energy and variance is updated before going to the next MC cycle number. Since the step size $\delta$ has to be manually tuned, it makes the brute force Metropolis not the most efficient.

\subsubsection{Importance Sampling}
\label{subsubsect:Importance}
Instead of having to manually choose the step size like in the brute force method, we want the step size to be determined by the trial wave function of the system. This introduces the Importance sampling where the main goal is to increase the acceptance rate such that we don't use unnecessary time on unwanted moves. This algorithm is based on the Fokker-Planck equation (\ref{eq:Fokker-Planck}) and the Langevin equation. In section \ref{subsect:F} we derived the drift/quantum force from the Fokker-Planck equation, which we will use in the Importance sampling algorithm soon.

To find the new position of the particles we use the Langevin equation
\begin{equation}
\label{eq:Langevin}
\frac{\partial \textbf{r}(t)}{\partial t}=DF(\textbf{r}(t))+\eta,
\end{equation}
where $D$ is the diffusion term, $F$ is the drift force and $\eta$ is a uniformly distributed stochastic variable for each dimension. The Langevin equation can be solved by using Euler's method:
\begin{equation}
\label{eq:Langevin_Euler}
\textbf{r}_{k+1}=\textbf{r}_k+DF(\textbf{r}_k)\Delta t+\xi\sqrt{\Delta t}
\end{equation}
$\xi$ is a Gaussian random variable and $\Delta t$ is the time step parameter to be chosen.

The Importance sampling algorithm implies that the transition probability can't be omitted any more. The Langevin equation provides with the new trial position, while the Fokker-Planck equation provides with the new transition probability on the form of Green's function:
\begin{equation}
\label{eq:Green}
G(\textbf{r}_k,\textbf{r}_{k+1},\Delta t)=\frac{1}{(4\pi D\Delta t)^{dN/2}}\exp\left(-\frac{(\textbf{r}_{k+1}-\textbf{r}_{k}-DF(\textbf{r}_k)\Delta t)^2}{4D\Delta t}\right)
\end{equation}
Equation \ref{eq:brute_accept_move} for the brute force Metropolis is now exchanged with the following Metropolis-Hastings algorithm condition for accepting the proposed move:
\begin{equation}
\label{eq:importance_accept_move}
r\leq\frac{G(\textbf{r}_k,\textbf{r}_{k+1},\Delta t)|\Psi_T(\textbf{r}_{k+1})|^2}{G(\textbf{r}_{k+1},\textbf{r}_{k},\Delta t)|\Psi_T(\textbf{r}_k)|^2}
\end{equation}
Here $r$ is a Gaussian distributed random number.

\subsection{Gradient Descent Method}
\label{subsect:Gradient}
For non-interacting bosons in a spherical trap, the optimal variational parameter was calculated analytically to be $\alpha=0.5$. This is not as simple when we introduce interaction between the bosons. So instead of having to manually find the best optimal variational parameter, which might be tough manually now, we want to find the optimal parameter in an iterative way. This can be done by introducing a gradient descent method to minimize the local energy with respect to $\alpha$. We start by choosing an initial $\alpha$ and find the next value as
\begin{equation}
\label{eq:grad_recursive}
\alpha_{k+1}=\alpha_k-\lambda\nabla_{\alpha}\cdot E_L(\alpha_k),
\end{equation}
where $\lambda$ is some small constant factor that we choose. When $\lambda$ is small enough we should see convergence to the desired minimum of $E_L$. At this minimum the derivative is zero and the method converges. To make sure that the minimum we get the global minimum and not a local minimum, we should run the algorithm several times with different initial guesses of $\alpha$. If they converge to the same minimum, then we have a one global minimum. We also have inserted two stopping criteria to end the computation, this is to free up some unnecessary computation. The gradient method stops if the gradient is smaller than a given tolerance or if the computations reaches a given max number of iterations suited for the computation.

\subsection{Blocking}
\label{subsect:Blocking}
To perform a statistical analysis (sect. \ref{subsect:Error_analysis}) of the Monte Carlo calculation we will use the blocking resampling method which is the better resampling method for bigger data sets. For smaller data sets we could have used resampling techniques like bootstrap or jackknife. The blocking method is well explained by \citet{flyvbjerg1989error}, and we will use a more improve blocking method developed by \citet{jonsson2018standard} in this project. 

The basic idea of the blocking method is that we start with a correlated data set which we will transform into an uncorrelated data set which we can use statistical analysis on. If we start with a data set $\textbf{X}=(X_1,X_2,\cdots,X_n)$. The data points $X_i$ in $\textbf{X}$ will be turned into new vectors with the mean of subsequent pairs for each $k$ blocking transformation. We define then $\textbf{X}_i$ recursively for all $1\leq i \leq d-1$:
\begin{align*}
(\textbf{X}_0)_k&=(\textbf{X})_k\\
(\textbf{X}_{i+1})_k&=\frac{1}{2}\left((\textbf{X}_i)_{2k-1}+(\textbf{X}_i)_{2k}\right)
\end{align*}
These new vectors are less correlated since the elements are further away from each other. By following the derivation by \citet{jonsson2018standard}, the variance of the sample mean of $\textbf{X}_k$ after $k$ blocking transformations is
\begin{equation}
\label{eq:blocking_var}
\sigma^2_b(\overline{\textbf{X}}_k)=\frac{\sigma_k^2}{n_k}+\frac{2}{n_k}\sum_{h=1}^{n_k-1}\left(1-\frac{h}{n_k}\right)\gamma_k(h)=\frac{\sigma_k^2}{n_k}+e_k\quad \text{if}\quad \gamma_k(0)=\sigma_k^2,
\end{equation}
where $\gamma_k$ is the auto-covariance of $\textbf{X}_k$ and $e_k$ is called the truncation error. \citet{flyvbjerg1989error} showed that the truncation error goes to zero as $k$ increases. After $e_k$ is small enough we should get a better estimation of the error in our results. This error should also be larger than we first calculated since it accounts for the correlations.

....................................
\section{Methods}
\label{sect:Methods}
-Parameters used is inserted in this section\\
-Blocking: use a pre-made Python program by \citet{jonsson2018standard} which we will modify a bit after what we want\\
-One-body density: Calculate the coordinate of every particle and place them into histograms for every MC cycle. Then we normalize by the number of particles and number of MC cycles. The radius sampling is first initialized after the equilibration time of the system has passed. This is to ensure that the system properties do not depend on the initial properties. 
\section{Results}
\label{sect:Results}
\section{Conclusion}
\label{sect:Conclusion}
..................................

\appendix
\section*{Appendix:}
\section{Non-Interacting Optimal Parameter}
\label{appendix:Optimal_alpha}
To find the optimal $\alpha$-parameter for non-interacting bosons for N particles in three dimensions with $\omega=1$ we take the partial derivative of the expected local energy in equation \ref{eq:local_E_nonint} with respect to the variational parameter:
\[\frac{\partial \langle E_L\rangle}{\partial \alpha}=0\]
\begin{align*}
\frac{\partial}{\partial \alpha}\left(\alpha dN+(-2\alpha^2+\frac{1}{2})\sum_{i=1}^{N}r_i^2\right)&=\frac{\partial}{\partial \alpha}\left(\alpha dN-\alpha^2dN+\frac{dN}{4}\right)=(dN-2\alpha dN)=0\\
\Rightarrow 2\alpha dN&=dN\\
\Rightarrow \alpha&=0.5
\end{align*}

\section{Derivation of The Interacting Laplacian}
\label{appendix:Laplacian}
Derivation of the Laplacian of the interacting trial wave function for a particle $k$:
\begin{align*}
\nabla_k^2\Psi_T(\textbf{r})&=\nabla_k^2\phi(\textbf{r}_k)\prod_{i\neq k}\phi(\textbf{r}_i)\exp\left(\sum_{j<m}u(r_{jm})\right)\\
&+ 2\nabla_k\phi(\textbf{r}_k)\prod_{i\neq k}\phi(\textbf{r}_i)\exp\left(\sum_{j<m}u(r_{jm})\right)\sum_{j\neq k}\nabla_ku(r_{kj})\\
&+ \prod_{i}\phi(\textbf{r}_i)\exp\left(\sum_{j<m}u(r_{jm})\right)\sum_{j\neq k}\nabla_ku(r_{kj})\sum_{i\neq k}\nabla_ku(r_{ki})\\
&+ \prod_{i}\phi(\textbf{r}_i)\exp\left(\sum_{j<m}u(r_{jm})\right)\sum_{j\neq k}\nabla_k^2u(r_{kj})
\end{align*}
The Laplacian divided by the trial wave function becomes:
\begin{align*}
\frac{\nabla_k^2\Psi_T(\textbf{r})}{\Psi_T(\textbf{r})}=\frac{\nabla_k^2\phi(\textbf{r}_k)}{\phi(\textbf{r}_k)}+2\frac{\nabla_k\phi(\textbf{r}_k)}{\phi(\textbf{r}_k)}\sum_{j\neq k}\nabla_ku(r_{kj})+\sum_{j\neq k}\nabla_ku(r_{kj})\sum_{i\neq k}\nabla_ku(r_{ki})+\sum_{j\neq k}\nabla_k^2u(r_{kj})
\end{align*}
Need now to compute $\nabla_k u(r_{kj})$ and $\nabla_k^2u(r_{kj})$:
\begin{align*}
\nabla_ku(r_{kj})&=u^{\prime}(r_{kj})\nabla_k r_{kj}=u^{\prime}(r_{kj})\nabla_k\sqrt{(\textbf{r}_k-\textbf{r}_j)^2}=u^{\prime}(r_{kj})\frac{\textbf{r}_k-\textbf{r}_j}{\sqrt{(\textbf{r}_k-\textbf{r}_j)^2}}=u^{\prime}(r_{kj})\frac{\Delta\textbf{r}_{kj}}{r_{kj}}\\
\nabla_k^2u(r_{kj})&=u^{\prime\prime}(r_{kj})\frac{\Delta\textbf{r}_{kj}}{r_{kj}}+u^{\prime}(r_{kj})\nabla_k\left(\frac{\Delta\textbf{r}_{kj}}{r_{kj}}\right)
=u^{\prime\prime}(r_{kj})+\frac{2}{r_{kj}}u^{\prime}(r_{kj})
\end{align*}
\begin{align*}
\Rightarrow \frac{\nabla_k^2\Psi_T(\textbf{r})}{\Psi_T(\textbf{r})}&= \frac{\nabla_k^2\phi(\textbf{r}_k)}{\phi(\textbf{r}_k)}+2\frac{\nabla_k\phi(\textbf{r}_k)}{\phi(\textbf{r}_k)}\left(\sum_{j\neq k}u^{\prime}(r_{kj})\frac{\Delta\textbf{r}_{kj}}{r_{kj}}\right)+\sum_{i,j\neq k}\frac{\Delta \textbf{r}_{ki}\Delta \textbf{r}_{kj}}{r_{ki}r_{kj}}u^{\prime}(r_{ki})u^{\prime}(r_{kj})\\
&+\sum_{j\neq k}\left(u^{\prime\prime}(r_{kj})+\frac{d-1}{r_{kj}}u^{\prime}(r_{kj})\right)
\end{align*}
\section{Derivation of Interaction Local Energy}
\label{appendix:local_E}
Inserting the gradient and Laplacian expressions for the single-particle function and the correlation function (eq. \ref{eq:eval_grad_one} to \ref{eq:eval_laplacian_int}) into the Laplacian divided by the trial wave function:
\begin{align}
\label{eq:H_int_deriv}
\frac{\nabla_k^2\Psi_T(\textbf{r})}{\Psi_T(\textbf{r})}&= \left(-2\alpha(d-1+\beta)+4\alpha^2(x_k^2+y_k^2+\beta^2z_k^2)\right)\nonumber\\
&+\left(-4\alpha(x_k\textbf{e}_x+y_k\textbf{e}_y+\beta z_k\textbf{e}_z)\right)\left(\sum_{j\neq k}\frac{a}{r_{kj}(r_{kj}-a)}\frac{\Delta\textbf{r}_{kj}}{r_{kj}}\right)\nonumber\\
&+\sum_{j\neq k}\left(\frac{a^2-2ar_{kj}}{r_{kj}^2(r_{kj}-a)^2}+\frac{d-1}{r_{kj}^2}\frac{a}{(r_{kj}-a)}\right)\nonumber\\
&+\left(\sum_{j\neq k}\frac{\Delta \textbf{r}_{kj}}{r_{kj}^2}\frac{a}{(r_{kj}-a)}\right)^2
\end{align}
The local energy for the interacting bosons:
\begin{align}
\label{eq:local_E_deriv}
E_L(\textbf{r})&=\sum_k^N\left(\frac{-\hbar^2}{2m}\frac{\nabla_k^2\Psi_T(\textbf{r})}{\Psi_T(\textbf{r})}+\frac{1}{\Psi_T(\mathbf{r})}V_{\text{ext}}(\textbf{r}_k)\Psi_T(\mathbf{r})\right)+\frac{1}{\Psi_T(\mathbf{r})}\sum_{j<k}^{N}V_{\text{int}}(\textbf{r}_j,\textbf{r}_k)\Psi_T(\mathbf{r})\nonumber\\
&=-\frac{\hbar^2}{2m}\sum_k^N\frac{\nabla_k^2\Psi_T(\textbf{r})}{\Psi_T(\textbf{r})}
+\frac{1}{2}m\sum_k^N[\omega^2(x_k^2+y_k^2) + \omega_z^2z_k^2]
+\sum_{j<k}^{N}V_{\text{int}}(\textbf{r}_j,\textbf{r}_k)
\end{align}

\bibliographystyle{plainnat}
\bibliography{myrefs}
\end{document}
